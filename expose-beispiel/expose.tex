\chapter{Exposé}

\section{Einleitung}

Der \emph{Deep Space 8K}\footnote{\url{http://www.aec.at/center/ausstellungen/deep-space/}} des Ars Electronica Center in Linz bietet mit seiner $16 \times 9$ Meter großen Projektionsfläche inklusive Positionstracking eine einzigartige Möglichkeit, Computerspiele zu realisieren. Diese Spiele verwenden keine klassischen Kontrollmechanismen wie Tastatur, Maus oder Gamepad sondern die SpielerInnen selbst steuern die Inhalte mit ihren Bewegungen. Darüber hinaus finden diese Spiele in einem halb-öffentlichen bis öffentlichen Raum statt, wodurch sich die Bestimmung der Zielgruppe sowie die Anzahl der spielenden Personen schwierig gestaltet. Diese Bachelorarbeit beschäftigt beleuchtet diese Problematik und stellt konkrete Lösungsvorschläge anhand eines Beispiels dar.

\section{Theoretischer Hintergrund und Stand der Forschung}

Large Public Display Games (LPD Games) sind Spiele, die auf großen, öffentlich einsehbaren Projektionsflächen dargestellt werden. Derlei Installationen finden sich etwa in Museen (wie dem Ars Electronica Center) oder auch auf öffentlichen Plätzen. Personen können diese Spiele in der Regel jederzeit sehen und auch aktiv an ihnen teilnehmen. Durch diese Öffentlichkeit ergeben sich nach \cite{Finke2008} drei Arten von Personengruppen, die am Spiel beteiligt sind: \emph{Actors} nehmen aktiv am Spielgeschehen teil, \emph{Specatators} verfolgen das Spielgeschehen aktiv und \emph{Bystanders} befindet sich lediglich in der Umgebung der öffentlichen Installation. Das Ziel ist es, dass Bystanders zur Spectators und Spectators zu Actors werden, also das Spiel aktiv spielen. Dieser Prozess soll dabei möglichst flißend von statten gehen und möglichst viele Personen umfassen. Ein derartiger Ansatz nennt sich \emph{Smooth Transition Gameplay} \cite{Hochleitner2013}, der anhand eines konkreten Anwendung demonstriert, wie dieser Übergang erreicht werden kann.

\section{Forschungsfrage}

% Interace vs Game Design?

\section{Methodik}

\section{Erwartete Ergebnisse}